\documentclass{article}
\usepackage{array}
\usepackage[numbers,square,sort]{natbib}

\begin{document}

\title{Configuring an Amazon EC2 environment to run scripts produced for emulab}
\author{Gavin Gresham}
\date{Apr 16, 2013}
\maketitle

\tableofcontents
\pagebreak

\section{Objective}
When I began experimenting with RUBBoS on Emulba, I quickly discovered that there were some limitations. There are a limited number of machines, and often times I ended up unable to run my experiments due to these limited resources. An alternative was needed.

This is were Amazon's Elastic Compute Cluster (EC2) has been remarkably useful. Virtual instances are sold by the hour, and are available in a multitude of hardware configurations. The challenge with these instances is that they begin as blank slates that lack any of the configuration that the RUBBoS scripts need to run. This is the objective of this project; To create an environment that allows EC2 to be used as a viable alternative to Emulab.
\section{Approach}
\subsection{Introduction to EC2}
Much like Emulab, Amazon Web Services (AWS) allow users to requisition virtual instances. The instances exist with various types of hardware, seen in Table \ref{ec2instancetypes}, and allow a wider variety of hardware to be tested. However, EC2 is just a small part of what AWS offers. It is also possible to use load balancers, cloud storage, relational and non-relation databases, and scalable communication methods. This suite of tools allows us to do some things that would be difficult to implement on the standard Emulab environment.

\begin{table*}[tb]
\centering
\begin{tabular}{|>{\raggedright}p{4cm}|c|c|}
\hline
Type & EC2 Compute Units & Memory\\\hline
M1 Small & 1 & 1.7GB\\\hline
M1 Medium & 2 & 3.75GB\\\hline
M1 Large & 4 & 7.5GB\\\hline
M1 Extra Large & 8 & 15GB\\\hline
M3\linebreak Extra Large & 13 & 15GB\\\hline
M3 Double\linebreak Extra Large & 26 & 30GB\\\hline
High-Memory\linebreak Extra Large & 6.5 & 17.1GB\\\hline
High-Memory\linebreak Double Extra Large & 13 & 34.2GB\\\hline
High-Memory\linebreak Quadruple Extra Large & 26 & 68.4GB\\\hline
High-CPU\linebreak Medium & 5 & 1.7GB\\\hline
High-CPU\linebreak Extra Large & 5 & 7GB\\\hline
Cluster Compute\linebreak Eight Extra Large & 88 & 60.5GB\\\hline
\end{tabular}
\caption{The hardware specifications for EC2 instances. "One EC2 Compute Unit provides the equivalent CPU capacity of a 1.0-1.2 GHz 2007 Opteron or 2007 Xeon processor". \cite{awsEC2Specs}}
\label{ec2instancetypes}
\end{table*}

\subsubsection{Types of Instances}
\subsubsection{On Demand}
These instances are available whenever you need them, and at a fixed cost. You gain the benefit of being able to acquire resources whenever you need them, but at the highest cost.
\subsubsection{Reserved}
Reserved instances have both an hourly cost, as well as an upfront cost. They have the same hardware specifications as "On Demand" instances, just with a different pricing model. Overall, you will save money if you maintain near 100\% utilization, but you cannot get the upfront back. These instances would be a poor choice for this type of project.
\subsubsection{Spot}
If you can be patient, these are likely the best type of instances for experiments. Once again, these share the same hardware specifications as "On Demand" and Reserved instances. The difference is that these instances are created by Amazon auctioning off idle compute power. This leads to a significantly reduced cost.
However, unlike other types of instances, Spot instances do not have a fixed price. You specify how much you are willing to pay per hour, and if the price rises above that your instance terminates. Despite the unpredictability of price, these are still the most logical choice for this project.
\subsection{Amazon Web Services JAVA API}
Much of this project focused on implementing various functions to interact with  the AWS SDK for Java. These methods deal with the creation of instances, assigning tags to those instances, and configuration elastic load balancers. The wrapper I have developed has been created specifically with Elba experiments in mind. It should allow for multiple experiments to run concurrently, and once they launch the experiments are not tied to application that launched them.
\subsection{Operating System Configuration}
There were several changes that were required to create a system image that would be compatible with RUBBoS scripts.
\section{Challenges}
\subsection{Cost}
While I do not know how Emulab is funded, it is important to note that AWS does have fees associated with its services. EC2 instances charge per hour used, seen in Table \ref{ec2instancepricing}, and larger experiments could cost a significant amount. This problem is tempered by Spot instances, which can cost 1/10th of traditional instances, but this only minimizes costs.

\begin{table}[bt]
\begin{tabular}{|>{\raggedright}p{4cm}|c|c|c|}
\hline
Type & On-Demand & Reserved$^1$ & Spot$^2$\\\hline
M1 Small & \$0.06 & \$0.014 (\$169) & \$0.007\\\hline
M1 Medium & \$0.12 & \$0.028 (\$338) & \$0.013\\\hline
M1 Large & \$0.24 & \$0.056 (\$676) & \$0.026\\\hline
M1 Extra Large & \$0.48 & \$0.112 (\$1352) &\$0.052\\\hline
M3\linebreak Extra Large & \$0.50 & \$.0123 (\$1489) & \$0.0575\\\hline
M3 Double\linebreak Extra Large & \$1.00 & \$0.246 (\$2978) & \$0.115\\\hline
High-Memory\linebreak Extra Large & \$0.41 & \$0.068 (\$789) & \$0.035\\\hline
High-Memory\linebreak Double Extra Large & \$0.82 & \$0.136 (\$1578) & \$0.07\\\hline
High-Memory\linebreak Quadruple Extra Large & \$1.64 & \$0.272 (\$3156) &\$0.14\\\hline
High-CPU\linebreak Medium & \$0.145 & \$0.036 (\$450) & \$0.018\\\hline
High-CPU\linebreak Extra Large & \$0.58 & \$0.144 (\$1800) & \$0.07\\\hline
Cluster Compute\linebreak Eight Extra Large & \$1.30 & \$0.297 (\$4060) & \$0.208\\\hline
\end{tabular}
\caption{The pricing for various types of EC2 instances.\cite{awsEC2pricing} \newline\newline
\footnotesize 1 The costs in parenthesis are yearly reservation costs.\newline
2 The average spot instance costs upon the time of writing. Prices can fluctuate significantly.}
\label{ec2instancepricing}
\end{table}

\section{Results}
\section{Future Development}
\subsection{Scaling}
As experiments scale, the sequential nature of server configuration in the scripts could cause signfinicant delays. To alleviate this problem, I propose a approach for simulataneously configuring all the servers.
A service exists within AWS called the Simple Notification Service (SNS) that can help accomplish this goal. SNS allows messages to be pushed to a group of subscribed clients. By setting up a HTTP servlet on each EC2 instance messages can be broadcast to every instance in an experiment. For example, if an experiment happens to have 50 apache servers all requiring compilation and configuration, the sequential setup time is both lengthy and on AWS costly. With SNS the control server could broadcast these commands that should be identical to every apache instance at once.
\subsection{Shared Drive}
\subsection{Generic Configuration}
While most of the AWSElbaAPI is generic, the parsing of experiment XML files and the commands run are specific to RUBBoS. Future development would either include a way for other developers to write custom parsers for their configurations, or inclusion of more parsers within the core app.
\subsection{Management Application}
A graphical representation of experiments are running and their state would be a great help to researcher I believe. I would like to create a GUI that would show running experiments, their status, and a menu for creating and loading experiments from XML files.
\section{Conclusion}

\bibliographystyle{plainnat}
\bibliography{bibliography}
\end{document}
